\documentclass{article}
\usepackage[utf8]{inputenc}
\usepackage{listings}
\lstset{
	language=Octave,
	frame=single,
	xleftmargin=.1\textwidth, xrightmargin=.1\textwidth
}
\usepackage{graphicx}
\usepackage{mathtools, nccmath}
\usepackage[T2A]{fontenc}
\usepackage[utf8]{inputenc}
\usepackage[russian]{babel}
\usepackage{amsmath}
\usepackage[left=2cm,right=2cm,top=2cm,bottom=2.1cm,bindingoffset=0cm]{geometry}
\usepackage{amsfonts}
\usepackage{amsmath}

\graphicspath{{/pic}}

\title{Задание № 3}
\author{Кокорин Илья, M3439}

\begin{document}
	\maketitle
	
	\section{Описание задания}
	
	Дано отношение с атрибутами StudentId, StudentName, GroupId, GroupName, CourseId, CourseName, LecturerId, LecturerName, Mark.
	
	\begin{enumerate}
		\item Найдите функциональные зависимости в данном отношении.
		\item Найдите все ключи данного отношения.
		\item Найдите неприводимое множество функциональных зависимостей для данного отношения.
	\end{enumerate}
	
	\section{Функциональные зависимости}
	
	В данном отношении есть следующие функциональные зависимости:
	
	\begin{enumerate}
		\item $StudentId \rightarrow StudentName, GroupId$ (студент имеет имя и обучается в группе)
		
		\item $GroupId \rightarrow GroupName$ (группа имеет название)
		
		\item $CourseId \rightarrow CourseName$ (предмет имеет название)
		
		\item $LecturerId \rightarrow LecturerName$ (лектор имеет имя)
		
		\item $GroupId, CourseId \rightarrow LecturerId$ (в группе предмет ведёт определённый человек, в 
		разных группах предмет могут вести разные люди)
		
		\item $StudentId, CourseId \rightarrow Mark$ (студент имеет оценку по предмету)
	\end{enumerate}

	\section{Ключи отношения}
	
	Будет находить ключ методом минимизации надключа.
	
	Очевидно, $(StudentId, StudentName, GroupId, GroupName, CourseId, CourseName, LecturerId, LecturerName, Mark)$ - надключ.
	
	Но не ключ, потому что $StudentId \rightarrow StudentName$ (По правилу расщепления ФЗ $StudentId \rightarrow StudentName, GroupId$ на $StudentId \rightarrow StudentName$ и $StudentId \rightarrow GroupId$). Следовательно, не выполнено требование минимальности по включению
	
	Убираем StudentName из надключа. 
	
	$(StudentId, GroupId, GroupName, CourseId, CourseName, LecturerId, LecturerName, Mark)$ - надключ.
	
	Аналогично поступаем с GroupName, CourseName, LecturerName (учитываем функциональные зависимости $GroupId \rightarrow GroupName, CourseId \rightarrow CourseName, LecturerId \rightarrow LecturerName$ )
	
	$(StudentId, GroupId, CourseId, LecturerId, Mark)$ - надключ.
	
	Но не ключ, так как $StudentId, CourseId \rightarrow Mark$. Следовательно, не выполнено требование минимальности по включению
	
	Убираем Mark из надключа. 
	
	$(StudentId, GroupId, CourseId, LecturerId)$ - надключ.
	
	Но не ключ, так как $GroupId, CourseId \rightarrow LecturerId$. Следовательно, не выполнено требование минимальности по включению.
	
	Убираем LecturerId из надключа. 
	
	$(StudentId, GroupId, CourseId)$ - надключ.
	
	Но не ключ, так как $StudentId \rightarrow GroupId$(По правилу расщепления ФЗ $StudentId \rightarrow StudentName, GroupId$ на $StudentId \rightarrow StudentName$ и $StudentId \rightarrow GroupId$). Следовательно, не выполнено требование минимальности по включению.
	
	Убираем GroupId из надключа. 
	
	$(StudentId, CourseId)$ - надключ.
	
	И в то же время ключ, так как отсутствует как функциональная зависимость $StudentId \rightarrow CourseId$, так и функциональная зависимость $CourseId \rightarrow StudentId$.
	
	Заметим, что любой другой порядок минимизации надключа привёл к точно такому же единственному ключу отношения.
	
	Ответ: $\{(StudentId, CourseId)\}$
	
	\section{Неприводимое множество}
	
	\subsection{Расщепление правых частей}
	
	Для начала расщепляем правые части всех ФЗ по правилу расщепления
	
	\begin{align*}
	S &=\\ 
	&\{ \\
	&StudentId \rightarrow StudentName,\\
	&StudentId \rightarrow GroupId,\\
	& GroupId \rightarrow GroupName,\\
	&CourseId \rightarrow CourseName,\\
	&LecturerId \rightarrow LecturerName,\\
	&GroupId, CourseId \rightarrow LecturerId,\\ 
	&StudentId, CourseId \rightarrow Mark\\
	&\}
	\end{align*}
	
	\subsection{Минимизация левых частей}
	
	Можно минимизировать левую часть в следующих ФЗ: $\{GroupId, CourseId \rightarrow LecturerId; StudentId, CourseId \rightarrow Mark\}$
	
	\begin{enumerate}
		\item
		Попробуем минимизировать левую часть $GroupId, CourseId \rightarrow LecturerId$
		
		\begin{enumerate}
			\item
			Попробуем исключить GroupId.
			
			$S' = S \setminus \{GroupId, CourseId \rightarrow LecturerId\} \cup \{CourseId \rightarrow LecturerId\} $
			
			Заметим, что $CourseId^+_S = \{CourseName\}$
			
			$CourseId^+_{S'} =  \{CourseName, LecturerId\}$
			
			То есть $CourseId^+_S \neq CourseId^+_{S'} $, то есть нельзя исключать GroupId.
			
			\item
			Попробуем исключить CourseId.
			
			$S' = S \setminus \{CourseId, GroupId \rightarrow LecturerId\} \cup \{GroupId \rightarrow LecturerId\} $
			
			Заметим, что $GroupId^+_S = \{GroupName\}$
			
			$GroupId^+_{S'} =  \{GroupName, LecturerId\}$
			
			То есть $GroupId^+_S \neq GroupId^+_{S'} $, то есть нельзя исключать CourseId.
		\end{enumerate}
	
		\item 
		Попробуем минимизировать левую часть $StudentId, CourseId \rightarrow Mark$
		
		\begin{enumerate}
			\item
			Попробуем исключить StudentId.
			
			$S' = S \setminus \{StudentId, CourseId \rightarrow Mark\} \cup \{CourseId \rightarrow Mark\} $
			
			Заметим, что $CourseId^+_S = \{CourseName\}$
			
			$CourseId^+_{S'} =  \{CourseName, Mark\}$
			
			То есть $CourseId^+_S \neq CourseId^+_{S'} $, то есть нельзя исключать StudentId.
			
			\item
			Попробуем исключить CourseId.
			
			$S' = S \setminus \{StudentId, CourseId \rightarrow Mark\} \cup \{StudentId \rightarrow Mark\} $
			
			Заметим, что $StudentId^+_S = \{StudentName, GroupId, GroupName\}$
			
			$StudentId^+_{S'} =  \{StudentName, GroupId, GroupName, Mark\}$
			
			То есть $StudentId^+_S \neq StudentId^+_{S'} $, то есть нельзя исключать CourseId.
		\end{enumerate}
	\end{enumerate}

	Значит, левая часть каждой ФЗ из S минимальна по включению.

	\subsection{Минимизация множества S}
	
	Будем по очереди пытаться исключить каждую ФЗ из множества S
	
	\begin{enumerate}
		\item 
		$S' = S \setminus \{StudentId \rightarrow StudentName\}$
		
		$StudentName \not\in StudentId^+_{S'} = \{GroupId, GroupName\}$
		
		\item 
		$S' = S \setminus \{StudentId \rightarrow GroupId\}$
		
		$GroupId \not\in StudentId^+_{S'} = \{StudentName\}$
		
		\item 
		$S' = S \setminus \{GroupId \rightarrow GroupName\}$
		
		$GroupName \not\in GroupId^+_{S'} = \emptyset$
		
		\item 
		$S' = S \setminus \{CourseId \rightarrow CourseName\}$
		
		$CourseName \not\in CourseId^+_{S'} = \emptyset$
		
		\item 
		$S' = S \setminus \{LecturerId \rightarrow LecturerName\}$
		
		$LecturerName \not\in LecturerId ^+_{S'} = \emptyset$
		
		\item 
		$S' = S \setminus \{GroupId, CourseId \rightarrow LecturerId\}$
		
		$LecturerId \not\in \{GroupId, CourseId\}^+_{S'} = \{GroupName, CourseName\}$
		
		\item 
		$S' = S \setminus \{StudentId, CourseId \rightarrow Mark\}$
		
		$Mark \not\in \{StudentId, CourseId\}^+_{S'} = \{StudentName, GroupId, GroupName, CourseName, LecturerId, LecturerName\}$
	\end{enumerate}

	Так как никакой элемент из S исключить нельзя, S минимален по включению.
	
	Значит, S -неприводимое множество ФЗ.
\end{document}