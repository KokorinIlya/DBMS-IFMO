\documentclass{article}
\usepackage[utf8]{inputenc}
\usepackage{listings}
\lstset{
	language=Octave,
	frame=single,
	xleftmargin=.1\textwidth, xrightmargin=.1\textwidth
}
\usepackage{graphicx}
\usepackage{mathtools, nccmath}
\usepackage[T2A]{fontenc}
\usepackage[utf8]{inputenc}
\usepackage[russian]{babel}
\usepackage{amsmath}
\usepackage[left=2cm,right=2cm,top=2cm,bottom=2.1cm,bindingoffset=0cm]{geometry}
\usepackage{amsfonts}
\usepackage{minted}
\usepackage{amssymb}
\usepackage{textcomp}

\graphicspath{{/pic}}
\DeclarePairedDelimiter{\nint}\lfloor\rfloor
\DeclarePairedDelimiter{\hint}\lceil\rceil

\def\ojoin{\setbox0=\hbox{$\bowtie$}%
  \rule[-.02ex]{.25em}{.4pt}\llap{\rule[\ht0]{.25em}{.4pt}}}
\def\leftouterjoin{\mathbin{\ojoin\mkern-5.8mu\bowtie}}
\def\rightouterjoin{\mathbin{\bowtie\mkern-5.8mu\ojoin}}
\def\fullouterjoin{\mathbin{\ojoin\mkern-5.8mu\bowtie\mkern-5.8mu\ojoin}}


\title{Приведение отношений в 5 НФ}
\author{Кокорин Илья, M3439}

\begin{document}
	\maketitle
	

\section{Songs}

\subsection{Функциональные зависимости}
\begin{enumerate}
	\item song\_id $\rightarrow$ name
	\item song\_id $\rightarrow$ text
	\item song\_id $\rightarrow$ duration
	\item song\_id $\rightarrow$ album\_id
	\item song\_id $\rightarrow$ artist\_id
\end{enumerate}

\subsection{Ключи}
Заметим, что $\{song\_id\}$ является ключом, и этот ключ единственный (так как все атрибуты определяются song\_id, любой другой надключ не будет минимальным по включению, поэтому не будет ключом)

\subsection{Нормальные формы}

\subsubsection{1 НФ}

Отношение уже находится в 1 НФ, так как в отношении нет повторияющихся групп, все атрибуты атомарны, а у отношения есть ключ

\subsubsection{2 НФ}

Отношение уже находится в 2 НФ, так как не имеет составных ключей.

\subsubsection{3 НФ}

Отношение уже находится в 3 НФ, так как не имеет никаких ФЗ, кроме зависимости всех атрибутов от ключа.

\subsubsection{НФБК}

Отношение находится в 3 НФ и не имеет перекрывающихся ключей $\Rightarrow$ находится в НФБК

\subsubsection{4 НФ}

По теореме Дейта-Фейгина, отношение находится в НФБК и существует простой ключ $\Rightarrow$ находится в 4 НФ

\subsubsection{5 НФ}

По теореме Дейта-Фейгина, отношение находится в 3 НФ и все ключи простые $\Rightarrow$ находится в 5 НФ

\section{Albums}

\subsection{Функциональные зависимости}
\begin{enumerate}
	\item album\_id $\rightarrow$ name
	\item album\_id $\rightarrow$ artist\_id
	\item album\_id $\rightarrow$ song\_id
	\item album\_id $\rightarrow$ release\_date
\end{enumerate}

\subsection{Ключи}
Заметим, что $\{album\_id\}$ является ключом, и этот ключ единственный (так как все атрибуты определяются album\_id, любой другой надключ не будет минимальным по включению, поэтому не будет ключом)

\subsection{Нормальные формы}

\subsubsection{1 НФ}

Отношение уже находится в 1 НФ, так как в отношении нет повторияющихся групп, все атрибуты атомарны, а у отношения есть ключ

\subsubsection{2 НФ}

Отношение уже находится в 2 НФ, так как не имеет составных ключей.

\subsubsection{3 НФ}

Отношение уже находится в 3 НФ, так как не имеет никаких ФЗ, кроме зависимости всех атрибутов от ключа.

\subsubsection{НФБК}

Отношение находится в 3 НФ и не имеет перекрывающихся ключей $\Rightarrow$ находится в НФБК

\subsubsection{4 НФ}

По теореме Дейта-Фейгина, отношение находится в НФБК и существует простой ключ $\Rightarrow$ находится в 4 НФ

\subsubsection{5 НФ}

По теореме Дейта-Фейгина, отношение находится в 3 НФ и все ключи простые $\Rightarrow$ находится в 5 НФ


\section{AlbumCovers}

\subsection{Функциональные зависимости}
\begin{enumerate}
	\item $album\_id, sequence\_number \rightarrow cover_path$
	\item $cover\_path \rightarrow album\_id, sequence\_number$
\end{enumerate}

\subsection{Ключи}

Заметим, что $\{album\_id, sequence\_number\}$ является ключом. Также ключом является $\{cover\_path\}$ Других ключей быть не может, так как  добавлять cover\_path к чему-то бесполезно, тк тогда ключ уже не будет минимален по включению, а минимизировать $\{album\_id, sequence\_number\}$ также нельзя.

\subsection{Нормальные формы}

\subsubsection{1 НФ}

Отношение уже находится в 1 НФ, так как в отношении нет повторияющихся групп, все атрибуты атомарны, а у отношения есть ключ

\subsubsection{2 НФ}

Отношение уже находится в 2 НФ, так как нет функционалиных зависимостей атрибутов от части составного ключа.

\subsubsection{3 НФ}

Отношение уже находится в 3 НФ, так как не имеет никаких ФЗ, кроме зависимости всех остальных атрибутов от ключа, следовательно, не имеет транзитивных зависимостей

\subsubsection{НФБК}

Отношение находится в 3 НФ и не имеет перекрывающихся ключей $\Rightarrow$ находится в НФБК

\subsubsection{4 НФ}

Не существует нетривиальных МЗ, не являющихся ФЗ

\subsubsection{5 НФ}
Единственный вариант декомпозиции на > 2 отношения это (album\_id, cover\_path), (cover\_path, sequence\_number), (album\_id, seqnence\_number). Это корректный вриант декомпозиции, тк у нас есть кольцевое ограничение: если у альбома A есть обложка с номером N, если файл P принадлежит альбому A и если файл P находится под номером N, то файл P является N-ной обложкой альбома A. В этой декомпозиции каждое $X_i$ надключ, следовательно, отношение находится в 5 НФ.


\section{Artists}

\subsection{Функциональные зависимости}
\begin{enumerate}
	\item artist\_id $\rightarrow$ name
\end{enumerate}

\subsection{Ключи}
Заметим, что $\{artist\_id\}$ является ключом, и этот ключ единственный (так как все атрибуты определяются artist\_id, любой другой надключ не будет минимальным по включению, поэтому не будет ключом)

\subsection{Нормальные формы}

\subsubsection{1 НФ}

Отношение уже находится в 1 НФ, так как в отношении нет повторияющихся групп, все атрибуты атомарны, а у отношения есть ключ

\subsubsection{2 НФ}

Отношение уже находится в 2 НФ, так как не имеет составных ключей.

\subsubsection{3 НФ}

Отношение уже находится в 3 НФ, так как не имеет никаких ФЗ, кроме зависимости всех атрибутов от ключа.

\subsubsection{НФБК}

Отношение находится в 3 НФ и не имеет перекрывающихся ключей $\Rightarrow$ находится в НФБК

\subsubsection{4 НФ}

По теореме Дейта-Фейгина, отношение находится в НФБК и существует простой ключ $\Rightarrow$ находится в 4 НФ

\subsubsection{5 НФ}

По теореме Дейта-Фейгина, отношение находится в 3 НФ и все ключи простые $\Rightarrow$ находится в 5 НФ


\section{ArtistPhotos}

\subsection{Функциональные зависимости}
\begin{enumerate}
	\item $artist\_id, sequence\_number \rightarrow photo\_path$
	\item $photo\_path \rightarrow artist\_id, sequence\_number$
\end{enumerate}

\subsection{Ключи}

Заметим, что $\{artist\_id, sequence\_number\}$ является ключом. Также ключом является $\{photo\_path\}$ Других ключей быть не может, так как  добавлять photo\_path к чему-то бесполезно, тк тогда ключ уже не будет минимален по включению, а минимизировать $\{artist\_id, sequence\_number\}$ также нельзя.

\subsection{Нормальные формы}

\subsubsection{1 НФ}

Отношение уже находится в 1 НФ, так как в отношении нет повторияющихся групп, все атрибуты атомарны, а у отношения есть ключ

\subsubsection{2 НФ}

Отношение уже находится в 2 НФ, так как нет функционалиных зависимостей атрибутов от части составного ключа.

\subsubsection{3 НФ}

Отношение уже находится в 3 НФ, так как не имеет никаких ФЗ, кроме зависимости всех остальных атрибутов от ключа, следовательно, не имеет транзитивных зависимостей

\subsubsection{НФБК}

Отношение находится в 3 НФ и не имеет перекрывающихся ключей $\Rightarrow$ находится в НФБК

\subsubsection{4 НФ}

Не существует нетривиальных МЗ, не являющихся ФЗ

\subsubsection{5 НФ}
Единственный вариант декомпозиции на > 2 отношения это (artist\_id, photo\_path), (photo\_path, sequence\_number), (artist\_id, seqnence\_number). Это корректный вриант декомпозиции, тк у нас есть кольцевое ограничение: если у артиста A есть фото с номером N, если файл P принадлежит артисту A и если файл P находится под номером N, то файл P является N-ным фото артистаы A. В этой декомпозиции каждое $X_i$ надключ, следовательно, отношение находится в 5 НФ.

\section{AlbumAuthors}

\subsection{ФЗ}
ФЗ нет

\subsection{Ключи}
${album\_id, artist\_id}$

\subsection{Нормальные формы}

\subsubsection{1 НФ}

Отношение уже находится в 1 НФ, так как в отношении нет повторияющихся групп, все атрибуты атомарны, а у отношения есть ключ

\subsubsection{2 НФ}

Отношение уже находится в 2 НФ, так как нет функционалиных зависимостей.

\subsubsection{3 НФ}

Отношение уже находится в 3 НФ, так как не имеет никаких ФЗ.

\subsubsection{НФБК}

Отношение находится в 3 НФ и не имеет перекрывающихся ключей $\Rightarrow$ находится в НФБК

\subsubsection{4 НФ}

Не имеет никаких МЗ, кроме тривиальных $\Rightarrow$ находится в 4 НФ

\subsubsection{5 НФ}
Имеет всего 2 атрибута $\Rightarrow$ декомпозиция на $> 2$ отношения невозможна. Декомпозировать на 2 отношения не имеет смысла, так как отношения находится в 4 НФ, а она лучшая с точки зрения разбиения на 2 отношения.

\section{SongInAlbums}

\subsection{ФЗ}
\begin{enumerate}
	\item $song\_id, album\_id \rightarrow position$
	\item $song\_id, position \rightarrow album\_id$
\end{enumerate}

\subsection{Ключи}
Ключами являются $\{song\_id, album\_id\}$ и $\{song\_id, position\}$. Других ключей размера 2 быть не может, так как $song\_id, position \not\rightarrow album\_id$. Ключа размера 1 быть не может, так как никакой атрибут не определяет остальные два. Ключ размера три может существовать только один, он является надключом двух ключей размера 2 (не минимален по включению).

\subsection{Нормальные формы}

\subsubsection{1 НФ}

Отношение уже находится в 1 НФ, так как в отношении нет повторияющихся групп, все атрибуты атомарны, а у отношения есть ключ

\subsubsection{2 НФ}

Отношение уже находится в 2 НФ, так нет функциональных зависимостей от части составного ключа.

\subsubsection{3 НФ}

Отношение уже находится в 3 НФ, так как каждый ключ непосредственно определяет единственный атрибут, не входящий в него $\Rightarrow$ нет транзитивной зависимости от ключа.

\subsubsection{НФБК}

Существует две нетривиальных ФЗ, в каждой из них левая часть является надключом.

\subsubsection{4 НФ}

Существует нетривиальная МЗ, не являющаяся ФЗ $album\_id \twoheadrightarrow song\_id | position$, при этом album\_id не является надключом.

Декомпозируем на два отношения: $(\underline{album\_id}, \underline{song\_id})$ и $(\underline{album\_id}, \underline{position})$. Каждая из них находится в 4 НФ, так как не имеет нетривиальных МЗ, не являющихся ФЗ, а ФЗ в них всего одна (ключевой атрибут определяет единственный неключевой)

\subsubsection{5 НФ}
Оба отношения содержат по два атрибута, нет нетривиальных МЗС, значит, отношения находятся в 5 НФ.

\subsection{Заключение}
Не будет декомпозировать до 4 НФ, так как в таком случае мы не будем знать, на какой позиции в альбоме какая песня стоит, следовательно, не сможем проиграть альбом в правильном порядке. Оставим в НФБК.

\section{SongAuthors}

\subsection{ФЗ}
ФЗ нет

\subsection{Ключи}
${song\_id, artist\_id}$

\subsection{Нормальные формы}

\subsubsection{1 НФ}

Отношение уже находится в 1 НФ, так как в отношении нет повторияющихся групп, все атрибуты атомарны, а у отношения есть ключ

\subsubsection{2 НФ}

Отношение уже находится в 2 НФ, так как нет функционалиных зависимостей.

\subsubsection{3 НФ}

Отношение уже находится в 3 НФ, так как не имеет никаких ФЗ.

\subsubsection{НФБК}

Отношение находится в 3 НФ и не имеет перекрывающихся ключей $\Rightarrow$ находится в НФБК

\subsubsection{4 НФ}

Не имеет никаких МЗ, кроме тривиальных $\Rightarrow$ находится в 4 НФ

\subsubsection{5 НФ}
Имеет всего 2 атрибута $\Rightarrow$ декомпозиция на $> 2$ отношения невозможна. Декомпозировать на 2 отношения не имеет смысла, так как отношения находится в 4 НФ, а она лучшая с точки зрения разбиения на 2 отношения.

\section{Users}

\subsection{ФЗ}
\begin{enumerate}
	\item $user\_id \rightarrow login$ 
	\item $user\_id \rightarrow pass\_hash\_with\_salt$ 
	\item $login \rightarrow user\_id$ 
	\item $login \rightarrow pass\_hash\_with\_salt$ 
\end{enumerate}

\subsection{Ключи}
$\{user\_id\}$ и $\{login\}$ являются ключами. Других ключей нет, так как все надключи размера 2 и 3 не минимальны по включению, а pass\_hash\_with\_salt не может являться ключом, так как не определяет ни одного атрибута.

\subsection{Нормальные формы}

\subsubsection{1 НФ}

Отношение уже находится в 1 НФ, так как в отношении нет повторияющихся групп, все атрибуты атомарны, а у отношения есть ключ

\subsubsection{2 НФ}

Отношение уже находится в 2 НФ, так как все ключи простые $\rightarrow$ не может быть ФЗ от части составного ключа.

\subsubsection{3 НФ}

Отношение уже находится в 3 НФ, так как неключевые атрибуты зависят только от ключа.

\subsubsection{НФБК}

Отношение находится в 3 НФ и не имеет перекрывающихся ключей $\Rightarrow$ находится в НФБК

\subsubsection{4 НФ}

По теореме Дейта-Фейинга, отношение находится в НФБК и существует простой ключ $\Rightarrow$ находится в 4 НФ

\subsubsection{5 НФ}
По теореме Дейта-Фейинга, отношение находится в 3 НФ и все ключи простые $\Rightarrow$ находится в 5 НФ

\section{UserAvatars}

\subsection{ФЗ}
\begin{enumerate}
	\item $user\_id, sequence\_number \rightarrow avatar$
\end{enumerate}

\subsection{Ключи}

Заметим, что $\{artist\_id, sequence\_number\}$ является ключом. Также ключом является $\{photo\_path\}$ Других ключей быть не может, так как  добавлять photo\_path к чему-то бесполезно, тк тогда ключ уже не будет минимален по включению, а минимизировать $\{artist\_id, sequence\_number\}$ также нельзя.

\subsection{Нормальные формы}

\subsubsection{1 НФ}

Отношение уже находится в 1 НФ, так как в отношении нет повторияющихся групп, все атрибуты атомарны, а у отношения есть ключ

\subsubsection{2 НФ}

Отношение уже находится в 2 НФ, так как нет функционалиных зависимостей атрибутов от части составного ключа.

\subsubsection{3 НФ}

Отношение уже находится в 3 НФ, так как не имеет никаких ФЗ, кроме зависимости всех остальных атрибутов от ключа, следовательно, не имеет транзитивных зависимостей

\subsubsection{НФБК}

Отношение находится в 3 НФ и не имеет перекрывающихся ключей $\Rightarrow$ находится в НФБК

\subsubsection{4 НФ}

Не существует нетривиальных МЗ, не являющихся ФЗ

\subsubsection{5 НФ}
Единственный вариант декомпозиции на > 2 отношения это (artist\_id, photo\_path), (photo\_path, sequence\_number), (artist\_id, seqnence\_number). Это корректный вриант декомпозиции, тк у нас есть кольцевое ограничение: если у артиста A есть фото с номером N, если файл P принадлежит артисту A и если файл P находится под номером N, то файл P является N-ным фото артиста A. В этой декомпозиции каждое $X_i $надключ, следовательно, отношение находится в 5 НФ.

\section{AlbumRatings}

\subsection{Функциональные зависимости}
\begin{enumerate}
	\item $album\_id, user\_id \rightarrow rating$
\end{enumerate}

\subsection{Ключи}

Ключом является $\{album\_id, user\_id\}$. Других ключей нет, так как эти два атрибута ничем функционально не определяются, но в совокупности определяют единственный оставшийся атрибут.

\subsection{Нормальные формы}

\subsubsection{1 НФ}

Отношение уже находится в 1 НФ, так как в отношении нет повторяющихся групп, все атрибуты атомарны, а у отношения есть ключ

\subsubsection{2 НФ}

Отношение уже находится в 2 НФ, так как не имеет ФЗ от части составного ключа.

\subsubsection{3 НФ}

Отношение уже находится в 3 НФ, так как не имеет никаких ФЗ, кроме зависимости всех атрибутов от ключа.

\subsubsection{НФБК}

Отношение находится в 3 НФ и не имеет перекрывающихся ключей $\Rightarrow$ находится в НФБК

\subsubsection{4 НФ}

Отношение находится в 4 НФ, так как не имеет нетривиальных МЗ, не являющихся ФЗ

\subsubsection{5 НФ}

	Найдём все нетривиальные зависимости соединений.
	
	Попытаемся разрезать только на 3 части, так как на 2 бесполезно, так как лучшая НФ в смысле разрезания на 2 части - 4 НФ, а на 4 части не хватит атрибутов.
	
	Очевидно, что имеет смысл включать в проекции только по два атрибута (три не имеет смысла, так как тогда получится тривиальная зависимость соединения).
	
	Так как операция соединения ассоциатитивна и коммутативна, у нас есть только один вариант разбиения на 3 проекции:
	
	\begin{enumerate}
		\item $album\_id, user\_id$
		\item $album\_id, rating$
		\item $user\_id, rating$
	\end{enumerate}

	Очевидно, что их соединение не даст исходное отношение. 
	
	Представим такое отношение $R: album\_id, user\_id, rating$:
	
	\begin{center}
		\begin{tabular}{ccc}
			album\_id: & user\_id,& rating \\
			1 & 1 & 4\\ 
			1 & 2& 5 \\
			2 & 1 & 5\\
			2 & 2 & 4 \\
		\end{tabular}
	\end{center}

	Тогда $\pi_{album\_id, user\_id}(R) = $
	
		\begin{center}
		\begin{tabular}{ccc}
			album\_id: & user\_id \\
			1 & 1\\
			1 & 2 \\
			2 & 1 \\
			2 & 2 \\
		\end{tabular}
	\end{center}

	Тогда $\pi_{album\_id, rating}(R) = $
	
	\begin{center}
		\begin{tabular}{ccc}
			album\_id & rating \\
			1 & 4\\
			1 & 5 \\
			2 & 5\\
			2 & 4\\
		\end{tabular}
	\end{center}

	Тогда $\pi_{user\_id, rating}(R) = $
	
	\begin{center}
		\begin{tabular}{ccc}
			user\_id & rating \\
			1 & 4\\
			2 & 5 \\
			1 & 5\\
			2 & 4\\
		\end{tabular}
	\end{center}


		$\pi_{album\_id, user\_id}(R) \bowtie \pi_{album\_id, rating}(R) =$
		
		\begin{center}
			\begin{tabular}{ccc}
				album\_id: & user\_id,& rating \\
				1 & 1 & 4\\
				1 & 2& 4 \\
				1 & 1 & 5\\
				1 & 2 & 5\\
				2 & 1 & 4\\
				2 & 1 & 5\\
				2 & 2 & 4\\
				2 & 2 & 5\\
			\end{tabular}
		\end{center}
	
	$(\pi_{album\_id, user\_id}(R) \bowtie \pi_{album\_id, rating}(R)) \bowtie  \pi_{user\_id, rating}(R) = $
	
	\begin{center}
		\begin{tabular}{ccc}
			GroupId: & CourseId,& LecturerId \\
			1 & 1 & 4\\
			1 & 2& 4 \\
			1 & 1 & 5\\
			1 & 2 & 5\\
			2 & 1 & 4\\
			2 & 1 & 5\\
			2 & 2 & 4\\
			2 & 2 & 5\\
		\end{tabular}
	\end{center}
	
	
	То есть $(\pi_{album\_id, user\_id}(R) \bowtie \pi_{album\_id, rating}(R)) \bowtie  \pi_{user\_id, rating}(R) \neq R$, так как в отношении появились лишние строки. То есть не сущесвует нетривиальных зависимостей соединений для отношения.
	
	Значит, отношение находится в 5 НФ.

\section{SongRatings}

\subsection{Функциональные зависимости}
\begin{enumerate}
	\item $song\_id, user\_id \rightarrow rating$
\end{enumerate}

\subsection{Ключи}

Ключом является $\{album\_id, user\_id\}$. Других ключей нет, так как эти два атрибута ничем функционально не определяются, но в совокупности определяют единственный оставшийся атрибут.

\subsection{Нормальные формы}

\subsubsection{1 НФ}

Отношение уже находится в 1 НФ, так как в отношении нет повторяющихся групп, все атрибуты атомарны, а у отношения есть ключ

\subsubsection{2 НФ}

Отношение уже находится в 2 НФ, так как не имеет ФЗ от части составного ключа.

\subsubsection{3 НФ}

Отношение уже находится в 3 НФ, так как не имеет никаких ФЗ, кроме зависимости всех атрибутов от ключа.

\subsubsection{НФБК}

Отношение находится в 3 НФ и не имеет перекрывающихся ключей $\Rightarrow$ находится в НФБК

\subsubsection{4 НФ}

Отношение находится в 4 НФ, так как не имеет нетривиальных МЗ, не являющихся ФЗ

\subsubsection{5 НФ}

с

\section{Playlists}

\subsection{Функциональные зависимости}
\begin{enumerate}
	\item playlist\_id $\rightarrow$ name
	\item playlist\_id $\rightarrow$ owner\_id

\end{enumerate}

\subsection{Ключи}
Заметим, что $\{playlist\_id\}$ является ключом, и этот ключ единственный (так как все атрибуты определяются playlist\_id, любой другой надключ не будет минимальным по включению, поэтому не будет ключом)

\subsection{Нормальные формы}

\subsubsection{1 НФ}

Отношение уже находится в 1 НФ, так как в отношении нет повторияющихся групп, все атрибуты атомарны, а у отношения есть ключ

\subsubsection{2 НФ}

Отношение уже находится в 2 НФ, так как не имеет составных ключей.

\subsubsection{3 НФ}

Отношение уже находится в 3 НФ, так как не имеет никаких ФЗ, кроме зависимости всех атрибутов от ключа.

\subsubsection{НФБК}

Отношение находится в 3 НФ и не имеет перекрывающихся ключей $\Rightarrow$ находится в НФБК

\subsubsection{4 НФ}

По теореме Дейта-Фейгина, отношение находится в НФБК и существует простой ключ $\Rightarrow$ находится в 4 НФ

\subsubsection{5 НФ}

По теореме Дейта-Фейгина, отношение находится в 3 НФ и все ключи простые $\Rightarrow$ находится в 5 НФ

\section{SongInPlaylists}

\subsection{Функциональные зависимости}

\begin{enumerate}
	\item $playlist\_id, position \rightarrow song\_id$
\end{enumerate}

\subsection{Ключи}

Ключом является $\{playlist\_id, position\}$. Других ключей нет, так как эти атрибуты ни от чего функционально не зависят.

\subsection{Нормальные формы}

\subsubsection{1 НФ}

Отношение уже находится в 1 НФ, так как в отношении нет повторияющихся групп, все атрибуты атомарны, а у отношения есть ключ

\subsubsection{2 НФ}

Отношение уже находится в 2 НФ, так как ни один неключевой атрибут не зависит от части составного ключа.

\subsubsection{3 НФ}

Отношение уже находится в 3 НФ, так как не имеет никаких ФЗ, кроме зависимости всех атрибутов от ключа.

\subsubsection{НФБК}

Отношение находится в 3 НФ и не имеет перекрывающихся ключей $\Rightarrow$ находится в НФБК

\subsubsection{4 НФ}

Нет нетривиальных МЗ, не являющихся ФЗ

\subsubsection{5 НФ}


	Найдём все нетривиальные зависимости соединений.
	
	Попытаемся разрезать только на 3 части, так как на 2 бесполезно, так как лучшая НФ в смысле разрезания на 2 части - 4 НФ, а на 4 части не хватит атрибутов.
	
	Очевидно, что имеет смысл включать в проекции только по два атрибута (три не имеет смысла, так как тогда получится тривиальная зависимость соединения).
	
	Так как операция соединения ассоциатитивна и коммутативна, у нас есть только один вариант разбиения на 3 проекции:
	
	\begin{enumerate}
		\item $playlist\_id, position$
		\item $position, song\_id$
		\item $song\_id, playlist\_id$
	\end{enumerate}

	Очевидно, что их соединение не даст исходное отношение. 
	
	Представим такое отношение $R: playlist\_id, position, song\_id$:
	
	\begin{center}
		\begin{tabular}{ccc}
			playlist\_id: & position,& song\_id \\
			1 & 1 & 1\\ 
			1 & 2& 2 \\
			2 & 1 & 2\\
			2 & 2 & 1 \\
		\end{tabular}
	\end{center}

	Тогда $\pi_{playlist\_id, position}(R) = $
	
		\begin{center}
		\begin{tabular}{ccc}
			playlist\_id: & position \\
			1 & 1\\
			1 & 2 \\
			2 & 1 \\
			2 & 2 \\
		\end{tabular}
	\end{center}

	Тогда $\pi_{position, song\_id}(R) = $
	
	\begin{center}
		\begin{tabular}{ccc}
			position & song\_id \\
			1 & 1\\
			2 & 2 \\
			1 & 2\\
			2 & 1\\
		\end{tabular}
	\end{center}

	Тогда $\pi_{song\_id, playlist\_id}(R) = $
	
	\begin{center}
		\begin{tabular}{ccc}
			song\_id & playlist\_id \\
			1 & 1\\
			1 & 2 \\
			2 & 2\\
			2 & 1\\
		\end{tabular}
	\end{center}


		$\pi_{playlist\_id, position}(R) \bowtie \pi_{position, song\_id}(R) =$
		
		\begin{center}
			\begin{tabular}{ccc}
				playlist\_id: & position,& song\_id \\
				1 & 1 & 1\\
				1 & 1 & 2 \\
				1 & 2 & 1\\
				1 & 2 & 2\\
				2 & 1 & 2\\
				2 & 1 & 2\\
				2 & 2 & 1\\
				2 & 2 & 2\\
			\end{tabular}
		\end{center}
	
	$(\pi_{playlist\_id, position}(R) \bowtie \pi_{position, song\_id}(R)) \bowtie  \pi_{song\_id, playlist\_id}(R) = $
	
		\begin{center}
				\begin{tabular}{ccc}
					playlist\_id: & position,& song\_id \\
					1 & 1 & 1\\
					1 & 1 & 2 \\
					1 & 2 & 1\\
					1 & 2 & 2\\
					2 & 1 & 2\\
					2 & 1 & 2\\
					2 & 2 & 1\\
					2 & 2 & 2\\
				\end{tabular}
			\end{center}
	
	
	То есть $(\pi_{playlist\_id, position}(R) \bowtie \pi_{position, song\_id}(R)) \bowtie  \pi_{song\_id, playlist\_id}(R) \neq R$, так как в отношении появились лишние строки. То есть не сущесвует нетривиальных зависимостей соединений для отношения.
	
	Значит, отношение находится в 5 НФ.

\end{document}